\documentclass{article}
\title{Requirement Specification}
\author{Internship Group 6}
\begin{document}
\maketitle
\noindent
\textbf{Project Description:}\\
A web-based platform for AURAK to manage, archive, and showcase university events. It includes support for event creation, scheduling, media uploads (images/videos), and long-term archiving.
\par\noindent\rule{\textwidth}{0.5pt}
\textbf{Functional Requirements:}
\begin{itemize}
    \item \textbf{User Registration and Authentication:} Users must be able to register, log in, and log out securely.
    \item \textbf{Event Creation:} Event managers can create events with details like title, description, date, time, and location.
    \item \textbf{Media Upload:} Event managers can upload images and videos related to events.
    \item \textbf{Event Scheduling:} Event managers can schedule events on a calendar.
    \item \textbf{Event Archiving:} Events can be archived for long-term storage.
    \item \textbf{Search Functionality:} Users can search for past events by keywords or dates.
\end{itemize}
\par\noindent\rule{\textwidth}{0.5pt}
\textbf{Non-Functional Requirements:}
\begin{itemize}
    \item \textbf{Performance:} The website needs to be responsive and able to handle up to 50 users concurrently.
    \item \textbf{Security:} User data must be encrypted and secure against unauthorized access.
    \item \textbf{Usability:} The interface should be intuitive and user-friendly, allowing users to navigate easily.
    \item \textbf{Scalability:} The system should be able to scale to accommodate future growth in user base and data volume.
    \item \textbf{Deployability:} The application should be deployable on the university's infrastructure or a cloud service.
    \end{itemize}
\par\noindent\rule{\textwidth}{0.5pt}
\end{document}